%\documentclass[12pt,twoside]{article}
\documentclass[conference]{IEEEtran}  % this should work with your LaTeX installation; else download extra package (www.ctan.org/pkg/ieeetran) or remove IEEE usage below

%%%%%%% Fill this out:
\newcommand{\trtitle}{Detecting Opinion Spammer Groups}
\newcommand{\titleshort}{NNs for Artificial Agents} % title for header:
\newcommand{\authorlastnames}{Adams, Jefferson, Washington} % alphabetical for seminars
\newcommand{\trcourse}{Knowledge Processing in Intelligent Systems: Practical Seminar}
\newcommand{\trgroup}{Knowledge Technology, WTM}
\newcommand{\truniversity}{Department of Informatics, University of Hamburg}
%%%%%%%%%%%%%%%%%%%%%%%%%%%%%%%%%%%%%%%%%%%%%%%%%%%%%%%%%%%%%
% Languages:

% If the thesis is written in English:
\usepackage[english]{babel} 						
\selectlanguage{english}

%%%%%%%%%%%%%%%%%%%%%%%%%%%%%%%%%%%%%%%%%%%%%%%%%%%%%%%%%%%%%
% Bind packages:
\usepackage{lipsum}

\usepackage{acronym}                    % Acronyms
\usepackage{algorithmic}								% Algorithms and Pseudocode
\usepackage{algorithm}									% Algorithms and Pseudocode
\usepackage{amsfonts}                   % AMS Math Packet (Fonts)
\usepackage{amsmath}                    % AMS Math Packet
\usepackage{amssymb}                    % Additional mathematical symbols
\usepackage{amsthm}
\usepackage{booktabs}                   % Nicer tables
%\usepackage[font=small,labelfont=bf]{caption} % Numbered captions for figures
\usepackage{color}                      % Enables defining of colors via \definecolor
\definecolor{uhhRed}{RGB}{254,0,0}		  % Official Uni Hamburg Red
\definecolor{uhhGrey}{RGB}{122,122,120} % Official Uni Hamburg Grey
\usepackage{fancybox}                   % Gleichungen einrahmen
\usepackage{fancyhdr}										% Packet for nicer headers
%\usepackage{fancyheadings}             % Nicer numbering of headlines

%\usepackage[outer=3.35cm]{geometry} 	  % Type area (size, margins...) !!!Release version
%\usepackage[outer=2.5cm]{geometry} 		% Type area (size, margins...) !!!Print version
%\usepackage{geometry} 									% Type area (size, margins...) !!!Proofread version
\usepackage{geometry} 	  % Type area (size, margins...) !!!Draft version
\geometry{a4paper,body={7.0in,9.1in}}

\usepackage{graphicx}                   % Inclusion of graphics
%\usepackage{latexsym}                  % Special symbols
\usepackage{longtable}									% Allow tables over several parges
\usepackage{listings}                   % Nicer source code listings
\usepackage{multicol}										% Content of a table over several columns
\usepackage{multirow}										% Content of a table over several rows
\usepackage{rotating}										% Alows to rotate text and objects
%\usepackage[hang]{subfigure}            % Allows to use multiple (partial) figures in a fig
%\usepackage[font=footnotesize,labelfont=rm]{subfig}	% Pictures in a floating environment
\usepackage{tabularx}										% Tables with fixed width but variable rows
\usepackage{url,xspace,boxedminipage}   % Accurate display of URLs

%%%%%%%%%%%%%%%%%%%%%%%%%%%%%%%%%%%%%%%%%%%%%%%%%%%%%%%%%%%%%
%Own packages
\usepackage{graphicx}
\usepackage{subfig}

%%%%%%%%%%%%%%%%%%%%%%%%%%%%%%%%%%%%%%%%%%%%%%%%%%%%%%%%%%%%%
% Configurationen:

\hyphenation{whe-ther} 									% Manually use: "\-" in a word: Staats\-ver\-trag

%\lstloadlanguages{C}                   % Set the default language for listings
\DeclareGraphicsExtensions{.pdf,.svg,.jpg,.png,.eps} % first try pdf, then eps, png and jpg
\graphicspath{{./src/}} 								% Path to a folder where all pictures are located
\pagestyle{fancy} 											% Use nicer header and footer

% Redefine the environments for floating objects:
\setcounter{topnumber}{3}
\setcounter{bottomnumber}{2}
\setcounter{totalnumber}{4}
\renewcommand{\topfraction}{0.9} 			  %Standard: 0.7
\renewcommand{\bottomfraction}{0.5}		  %Standard: 0.3
\renewcommand{\textfraction}{0.1}		  	%Standard: 0.2
\renewcommand{\floatpagefraction}{0.8} 	%Standard: 0.5

% Tables with a nicer padding:
\renewcommand{\arraystretch}{1.2}

%%%%%%%%%%%%%%%%%%%%%%%%%%%%
% Additional 'theorem' and 'definition' blocks:
\theoremstyle{plain}
\newtheorem{theorem}{Theorem}[section]
%\newtheorem{theorem}{Satz}[section]		% Wenn in Deutsch geschrieben wird.
\newtheorem{axiom}{Axiom}[section] 	
%\newtheorem{axiom}{Fakt}[chapter]			% Wenn in Deutsch geschrieben wird.
%Usage:%\begin{axiom}[optional description]%Main part%\end{fakt}

\theoremstyle{definition}
\newtheorem{definition}{Definition}[section]

%Additional types of axioms:
\newtheorem{lemma}[axiom]{Lemma}
\newtheorem{observation}[axiom]{Observation}

%Additional types of definitions:
\theoremstyle{remark}
%\newtheorem{remark}[definition]{Bemerkung} % Wenn in Deutsch geschrieben wird.
\newtheorem{remark}[definition]{Remark} 

%%%%%%%%%%%%%%%%%%%%%%%%%%%%
% Provides TODOs within the margin:
\newcommand{\TODO}[1]{\marginpar{\emph{\small{{\bf TODO: } #1}}}}

%%%%%%%%%%%%%%%%%%%%%%%%%%%%
% Abbreviations and mathematical symbols
\newcommand{\modd}{\text{ mod }}
\newcommand{\RS}{\mathbb{R}}
\newcommand{\NS}{\mathbb{N}}
\newcommand{\ZS}{\mathbb{Z}}
\newcommand{\dnormal}{\mathit{N}}
\newcommand{\duniform}{\mathit{U}}

\newcommand{\erdos}{Erd\H{o}s}
\newcommand{\renyi}{-R\'{e}nyi}
\usepackage{graphicx}

% correct bad hyphenation here
\hyphenation{}

%%%%%%%%%%%%%%%%%%%%%%%%%%%%%%%%%%%%%%%%%%%%%%%%%%%%%%%%%%%%%
% Document:
\begin{document}

\title{\trtitle}
\renewcommand{\headheight}{14.5pt}

%\fancyhead{}
%\fancyhead[LE]{  }
\fancyhead[LO]{\slshape \authorlastnames}
%\fancyhead[RE]{}
\fancyhead[RO]{ \slshape \titleshort}

% author names and affiliations
% use a multiple column layout for up to three different
% affiliations
\author{
\IEEEauthorblockN{Lars Stelzer}
\IEEEauthorblockA{lars.stelzer@studium.uni-hamburg.de}
\IEEEauthorblockA{7346178}
\and
\IEEEauthorblockN{Niklas von Boguszewski}
\IEEEauthorblockA{nvboguszewski@googlemail.com}
\IEEEauthorblockA{6790872}
\and
\begin{tabular}{lr}% 
	\trcourse\\ 
	\trgroup,
	\truniversity
\end{tabular}

}


       


% make the title area
\maketitle

\begin{abstract}
This paper aims to detect opinion spammer groups on the Yelp dataset.



\end{abstract}

\IEEEpeerreviewmaketitle

\section{Introduction}
\label{sec:introduction}
This chapter shows the motivation of why this topic is important. 




\section{Related Work}
\label{sec:related_work}

In this section we show important work which has been made by other researchers like 
\cite{mukherjee2013spotting} and \cite{choo2015detecting}.

In this section we also want to highlight why this approach can be beneficial for this research area. 

\section{Yelps Review System}
\label{sec:Yelps review_system}

In this section we shortly describe how Yelps review system works. 

\section{Dataset}
\label{sec:dataset}

In this section we shortly describe the Yelp dataset.

\section{Feature}
\label{sec:feature}

In this section we descripe approximately 3 features we want to experiment with.  

\section{Model}
\label{sec:model}

In this section we describe our model/graph which aims to find opinion spamming groups.

\section{Results/Analysis}
\label{sec:analysis}

In this section we show our plots and describe the results/interpretation. 

\section{Conclusion}
\label{sec:concl}

\section{Experimental plots}
\label{sec:plots}

200.000 rows of the review json

\graphicspath{{plots/}}


\begin{figure}[H]
  \centering
  \subfloat[Positive]{\includegraphics[width=0.2\textwidth]{group_size}\label{fig:f1}}
  \hfill
  \subfloat[Negative]{\includegraphics[width=0.2\textwidth]{group_size_negative}\label{fig:f2}}
  \caption{Groups}
\end{figure}


\begin{figure}[H]
  \centering
  \subfloat[Positive]{\includegraphics[width=0.2\textwidth]{group_activity}\label{fig:f1}}
  \hfill
  \subfloat[Negative]{\includegraphics[width=0.2\textwidth]{group_activity_negative}\label{fig:f2}}
  \caption{Activity}
\end{figure}

\begin{figure}[H]
  \centering
  \subfloat[Positive]{\includegraphics[width=0.2\textwidth]{len_review}\label{fig:f1}}
  \hfill
  \subfloat[Negative]{\includegraphics[width=0.2\textwidth]{len_review_negative}\label{fig:f2}}
  \caption{Length}
\end{figure}

\begin{figure}[H]
  \centering
  \subfloat[Positive]{\includegraphics[width=0.2\textwidth]{stars_given}\label{fig:f1}}
  \hfill
  \subfloat[Negative]{\includegraphics[width=0.2\textwidth]{stars_given_negative}\label{fig:f2}}
  \caption{Stars}
\end{figure}



% insert your bibliographic references into the bib.bib file
\bibliographystyle{plain}
\addcontentsline{toc}{section}{Bibliography}% Add to the TOC
\bibliography{bib}
\end{document}
